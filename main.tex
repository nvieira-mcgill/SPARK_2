%\pdfoutput=1 %for arXiv submission
%\documentclass[iop,apj]{emulateapj}
\documentclass[twocolumn, twocolappendix]{aastex63}
\usepackage{hyperref}
\usepackage{amsmath,amstext}
\usepackage[T1]{fontenc}
%\usepackage{apjfonts} 
\usepackage{graphicx}
\usepackage[figure,figure*]{hypcap}
%\usepackage[fleqn]{amsmath}
\usepackage{multirow}
\usepackage{comment}
\usepackage{nicefrac}


\usepackage[]{algorithm2e} % algorithms


\renewcommand*{\sectionautorefname}{Section} %for \autoref
\renewcommand*{\subsectionautorefname}{Section} %for \autoref

%%% software mentioned again and again
\def\SPARK{\texttt{SPARK}}
\def\TARDIS{\texttt{TARDIS}}
\def\AUTOSTRUCTURE{\texttt{AUTOSTRUCTURE}}
\def\approxposterior{\texttt{approxposterior}}

%%% other useful commands
\def\citneeded{\textcolor{red}{\textbf{(citation needed)}}}
\newcommand\redbf[1]{\textbf{\textcolor{red}{#1}}}

\newcommand{\crf}[1]{{\color{violet} RF: #1}} %comment from Rodrigo
\usepackage{soul}
\newcommand{\remove}[1]{{\color{red} \st{#1}}}
\newcommand{\addtext}[1]{{\color{blue} #1}}

\def\lbol{${L}_{\rm bol}$}
\def\ledd{${L}_{\rm Edd}$}
\def\lsun{${L}_{\odot}$}
\def\sun{$_{\odot}$}
\def\lx{${L}_{\rm X}$}
\def\fx{${f}_{\rm X}$}
\def\f28{${f}_{2-8{\rm keV}}$}
\def\fr{${f}_R$}
\def\fo{f$_{opt}$}
\def\fxfr{${f}_X$/${f}_R$}
\def\fratio{${f}_X$/${f}_R$} 
\def\fxfopt{${f}_X$/${f}_{opt}$}
\def\fxfV{${f}_X$/${f}_V$}
\def\nh{N$_{\rm H}$}
\def\rc{r$_c$}
\def\mass{${\cal M}$}
\def\msun{${\cal M}_{\odot}$}
\def\mearth{${\cal M}_{\oplus}$}
\def\sun{$_{\odot}$}
\def\mdot{$\dot{\cal M}$}

\def\ergs{erg s$^{-1}$}
\def\ergscm2{erg s$^{-1}$ cm$^{-2}$}
\def\yr-1{yr$^{-1}$}
\def\kms{km s$^{-1}$}
\def\persec{\,\hbox{s}^{-1}}
\def\percc{\,\hbox{cm}^{-3}}
\def\persqcm{\,\hbox{cm}^{-2}}
\def\mic{\,\mu \hbox{m}}

\def\eg{{\it e.g.}}
\def\et{{\it et al.}}
\def\ie{{\it i.e.}}
\def\cf{{\it cf.}}

\def\a{$\&$}
\def\x{$\times$}
\def\about{$\sim$}
\def\simlt{\buildrel{<}\over \sim}
\def\simgt{\buildrel{>}\over \sim}
\def\simeq{\sim \over $=$}
\def\simgreat{\buildrel{>}\over \sim}
\def\simlt{$\la$}
\def\simgt{$\ga$}
\def\half{${\textstyle{1\over2}}$}
\def\thalf{{\textstyle{ 3\over 2}}}
\def\subr #1{_{{\rm #1}}}
\def\w{$\omega$}
\def\sig{$\sigma$}

\def\deg{$^{\rm o}$}
%\def\asec{$''$} 
\def\asec{\ifmmode^{\prime\prime}\else$^{\prime\prime}$\fi}
\def\amin{$'$}
\def\spt{$\buildrel{\prime\prime}\over .$}
\def\secspt{$\buildrel{\prime\prime}\over .$}
\def\minspt{$\buildrel{\prime}\over .$}
\def\magspt{$\buildrel{\rm m}\over .$}

%\def\Chandra{{\it Chandra}}
%\def\XMM{XMM {\it Newton}}
%\def\Rosat{{\it Rosat}}
%\def\Einstein{{\it Einstein}}
%\def\HST{{\it HST}}
%\def\Swift{{\it Swift}}
%\def\Nustar{{\it NuSTAR}}

\hypersetup{linkcolor=magenta}

\submitjournal{ApJ}
%\received{(ApJ) September 14, 2022}
%\accepted{on December 22, 2022}

\shorttitle{multi-component ejecta for the GW170817 kilonova} 
\shortauthors{Vieira {\it et al.}}

\begin{document}

\title{No Need for Multiple Components to fit the Spectra of the GW170817 Kilonova --- Spectroscopic $r$-Process Abundance Retrieval for Kilonovae II}

\correspondingauthor{Nicholas~Vieira}
\email{nicholas.vieira@mail.mcgill.ca}

\author[0000-0001-7815-7604]{Nicholas~Vieira}
\affil{Trottier Space Institute at McGill and Department of Physics, McGill University, 3600 rue University, Montreal, Qu{\'e}bec, H3A 2T8, Canada}

\author[0000-0001-8665-5523]{John~J.~Ruan}
\affil{Department of Physics and Astronomy, Bishop's University, 2600 rue College, Sherbrooke, Qu{\'e}bec, J1M 1Z7, Canada}

\author[0000-0001-6803-2138]{Daryl Haggard}
\affil{Trottier Space Institute at McGill and Department of Physics, McGill University, 3600 rue University, Montreal, Qu{\'e}bec, H3A 2T8, Canada}

\author[0000-0001-8921-3624]{Nicole Ford}
\affil{Trottier Space Institute at McGill and Department of Physics, McGill University, 3600 rue University, Montreal, Qu{\'e}bec, H3A 2T8, Canada}

% \author[0000-0001-7081-0082]{Maria~R.~Drout}
% \affil{Department of Astronomy and Astrophysics, University of Toronto, 50 St. George St., Toronto, Ontario, M5S 3H4, Canada}

% \author[0000-0003-4619-339X]{Rodrigo Fern{\'a}ndez}
% \affil{Department of Physics, University of Alberta, Edmonton, Alberta, T6G 2E1, Canada}

% \author{N.~R.~Badnell}
% \affil{Department of Physics, University of Strathclyde, Glasgow, G4 0NG, UK}



\begin{abstract}
x
\end{abstract}
\keywords{nuclear abundances, r-process, radiative transfer simulations, spectral line identification}

% ====================================================

%%% === SECTION 1 === %%%
%% INTRODUCTION %%
\section{Introduction}\label{sec:intro}

\begin{itemize}

    \item The $r$-process, its connection to neutron star mergers; connection to GW170817

    \item \SPARK~paper 1: technique, key results
    
    \item Single versus multi-component models of kilonova ejecta
    
    \item Goal of this paper: fit the later epochs to understand the time evolution of the abundances and determine whether multiple ejecta components are needed to fit the spectra
    
\end{itemize}

%%% === SECTION 2 === %%%
%% METHODS %%

\section{Methods}\label{sec:methods}

%%% SUBSECTION 2.1
\subsection{Spectroscopic $r$-Process Abundance Retrieval for Kilonovae (\textsc{SPARK})}\label{ssc:spark-summary}

We briefly describe our tool, Spectroscopic $r$-Process Abundance Retrieval for Kilonovae (\SPARK), and refer the reader to \cite{vieira23} for more detail. \SPARK is designed as a modular inference engine for extracting key parameters of kilonovae spectra, determining the element-by-element abundance pattern of the ejecta, and associating features in the spectrum with particular species. 

In \SPARK, we couple the 1D \TARDIS~(\citealt{kerzendorf14})~radiative transfer code to the approximate posterior estimation scheme of \approxposterior~(\citealt{fleming18,fleming20}). We generate a set of synthetic spectra, each parameterized by some $\theta_i$ including a luminosity, density, inner/outer computational boundary velocities, and three key parameters which determine the abundances in the ejecta: the electron fraction $Y_e$, expansion velocity $v_{\mathrm{exp}}$, and specific entropy per nucleon $s / k_{\mathrm{B}}$. These parameters describe different abundances from the nuclear reaction network calculations of \cite{wanajo18}. We write down a likelihood function using the full formalism of \cite{czekala15} for inference on spectroscopic data. Because of the considerable computational cost of spectral synthesis with \TARDIS, we do not use more common methods such as Markov chain Monte Carlo (MCMC) or nested sampling for inference---rather, we introduce a Gaussian Process (GP) surrogate for the posterior $L_p (\theta)$ and employ Bayesian Active Posterior Estimation (BAPE; \citealt{kandasamy17}). BAPE is a form of active learning in which we maximize an acquisition function with terms including both the mean $\mu(\theta)$ and variance $\sigma^2(\theta)$ of the GP. This acquisition function thus balances exploration (of the parameter space) and exploitation (sampling around the peak of the posterior). The GP is iteratively re-trained as new points $(\theta, L_p(\theta))$ are added to a training set and converges to an approximation of the posterior. 

In all, inference is dramatically accelerated, and we obtain (among the other kilonova parameters) the $Y_e$, $v_{\mathrm{exp}}$, $s / k_{\mathrm{B}}$ which best describe the ejecta with relatively few forward model evaluations. In \cite{vieira23}, we fit the VLT/X-shooter spectrum of the GW170817 kilonova at 1.4 days post-merger (\citealt{pian17, smartt17}) with a baseline of 1500 Latin Hypercube samples + 1140 BAPE active learning samples. This is a factor of $\sim 10^3$ fewer samples than might be required with a standard MCMC for a similar 6-dimensional fit.


%%% SUBECTION 2.2
\subsection{Multi-component, stratified ejecta with \textsc{TARDIS}}\label{ssc:multi-component-TARDIS}

\begin{itemize}

    \item In \cite{vieira23}, we model the kilonova ejecta as a single shell with a uniform abundance pattern. The plasma in this shell is also described by a single temperature and mass/electron density. This configuration is fully described by the luminosity at the outer boundary $L_{\mathrm{outer}}$, (which in fact sets the temperature at the inner boundary), the normalization in the density power law $\rho_0$, the inner and outer boundary velocities $v_{\mathrm{inner}}$ and $v_{\mathrm{outer}}$, and three parameters which set the abundances: electron fraction $Y_e$, expansion velocity $v_{\mathrm{exp}}$, and specific entropy $s / k_{\mathrm{B}}$. A single-component fit is thus 7-dimensional, unless one or more of the parameters are fixed, \eg, we fix $v_{\mathrm{outer}} = 0.35c$ in our fit to the 1.4 day spectrum. This setup can describe a single ejecta component, \eg, the tidal ejecta mostly confined to the equatorial plane, an outflow from a remnant accretion disk, or the squeezed polar ejecta from the collision interface during the merger. It may also describe a kilonova in which one component significantly dominates (by mass or by the strength of the absorption/emission features) over the other(s). 
    
    \item Here, we implement multi-component, stratified ejecta. \TARDIS~allows for ejecta composed of stratified shells, each with a specific temperature, density, and plasma conditions. In this configuration, each shell can have a specific abundance pattern. We use 10 shells in all runs unless indicated otherwise. We also employ multiple (30) \TARDIS~iterations when generating synthetic spectra in this configuration. At each iteration, the plasma conditions are updated, and converge towards an ejecta where the outer boundary emitted luminosity matches the user-requested $L_{\mathrm{outer}}$. 
    
    \item We begin with a simple two-component ejecta. As with our single-component model, this two-component model is described by an outer boundary luminosity $L_{\mathrm{outer}}$ and $\rho_0$. Each of the components is then described by inner and outer boundary $v_{\mathrm{inner}}$ and $v_{\mathrm{outer}}$ and a $Y_e$, $v_{\mathrm{exp}}$, and $s / k_{\mathrm{B}}$. Multi-component fits are thus $2 + 5 + 5 = 12$-dimensional. The components necessarily overlap in physical space because \TARDIS~cannot simulate a gap between them. The abundance in each shell is then determined by the component(s) which are in a given shell. For shells where there is overlap between two components, the abundance is taken as a sum of the two abundance patterns.
    
    \item We demonstrate the difference between single- and multi-component ejecta in Figure X. Multiple components allow for additional complexity in the spectral synthesis.\footnote{See \cite{kawaguchi20} for a description of the diversity of kilonovae which may be produced when multiple components are present.} In particular, even in 1D with \TARDIS, we can produce the effect of re-processing, where emission from one component is absorbed and reemitted/scattered by another. We can also produce the effect of ``lanthanide curtaining'', in which some outermost lower-$Y_e$ ejecta containing the lanthanides masks an inner, bluer, lighter-element ejecta due to the considerable opacity of the lanthanides in the near-UV and optical. Some kilonova spectra may be better-described by these multi-component ejecta models. In \cite{vieira23}, we find that the 1.4 day spectrum of the GW170817 kilonova is well-described by a single-component ejecta with $Y_e \sim 0.3$. However, at later epochs, as the ejecta expands and becomes more optically thin, the photosphere recedes into the ejecta and we may unmask additional components which were hidden at early times. Light curve modelling (\eg, \citealt{villar17}) has shown that the kilonova may indeed be better-described by multiple components of different opacities, and some spectral modelling (\eg, \citealt{kasen17}) also invokes multiple components. This motivates our introduction of multi-component ejecta into \SPARK. We fit the 1.4, 2.4, and 3.4 day spectra of the GW170817 kilonova with both single- and multi-component models to assess the need for multiple components.

\end{itemize}



%%% SUBSECTION 2.3
\subsection{Inference setup}\label{ssc:inference-setup}

\begin{itemize}

    \item All \texttt{approxposterior}/BAPE hyperparameters and optimizers used here are the same as those used in \cite{vieira23}. We again produce a baseline run of $m_{0} = 1500$ Latin Hypercube sampled points at the beginning of each \SPARK~run. However, the parameter space allowed by our priors differs for the 1.4, 2.4, and 3.4 day fits. Table~\ref{tab:priors-single} includes our priors for each fit. All priors are uniform. The bounds on the density $\rho_0$ and abundance-setting parameters $Y_e$, $v_{\mathrm{exp}}$, and $s$ are the same at all epochs. The priors differ in the bounds on the luminosity, as expected given the cooling of the ejecta over time as it expands. We further allow for wider priors on the inner and outer boundary velocities for the fits at later epochs. In \cite{vieira23}, we fixed $v_{\mathrm{outer}} = 0.35c$ during our 1.4 day fit but noted that we observed similar results for $v_{\mathrm{outer}}$ in the range $0.35 - 0.38c$. Here, we allow for greater flexibility in our fits to the later epochs. \redbf{Mention/discuss: Does this complicate the inference? Is there a degeneracy between the inner and outer boundary velocities?}
    
    \item Our priors for our multi-component fits are given in Table~\ref{tab:priors-multi}. These are identical at all epochs. The priors for the two components in a given fit are also identical, except that the ``red'' component has $Y_{e,\mathrm{red}} \in [0.01, 0.45]$ while the ``blue'' has $Y_{e,\mathrm{blu}} \in [0.05, 0.50]$. We use the terms red and blue loosely to simplify our discussion but note that the priors do not enforce $Y_{e,\mathrm{red}} < Y_{e,\mathrm{blu}}$. The priors enforce $v_{\mathrm{inner,red}} < v_{\mathrm{outer,red}}$ and $v_{\mathrm{inner,blu}} < v_{\mathrm{outer,blu}}$. We also ensure some overlap between the two components, \ie, $v_{\mathrm{outer,red}} > v_{\mathrm{inner,blu}}$ or $v_{\mathrm{outer,blu}} > v_{\mathrm{inner,red}}$. \redbf{Finally, we also enforce that there is some non-overlap between the two components, \ie, one component does not completely envelop another.} To allow for even sampling of parameter space with this complex conditional prior, we use constrained Latin Hypercube Sampling (cLHS; \citealt{petelet09}).
    
    \item When attempting to fit the 3.4 day spectrum, we found that the fit converged to a blackbody with little to no absorption in order to fit the continuum of the spectrum. Since we are interested in determining the species involved in the absorption, and the employed observed line list is likely to be especially incomplete at the shortest wavelengths, we prioritize fitting the $\geqslant 6000$~\AA~region of the spectrum. The crudest way to do this would be to simply mask the shorter wavelengths during the fit. We instead use the full \cite{czekala15} formalism for introducing local covariance features into the likelihood during spectroscopic inference. We introduce a Gaussian feature with amplitude \redbf{$a_{\mathrm{L}}^2 = 10^{-34} (\mathrm{erg~s^{-1}~cm^{-2}}$~\text{\AA}${}^{-1})^{2}$, central wavelength $\mu_{\mathrm{L}} = 3000$~\AA, and variance $\sigma_{\mathrm{L}}^2 = 2500$~\AA${}^2$} into the covariance kernel. This technique effectively down-weights this region of the spectrum during the computation of the likelihood, but does not completely ignore it. For comparison, the amplitude of the global covariance structure is $a_{\mathrm{G}} = 10^{-34} (\mathrm{erg~s^{-1}~cm^{-2}}$~\text{\AA}${}^{-1})^{2}$ in all fits.

\end{itemize}


\begin{deluxetable}{c|ccc}
\centering
\tablecaption{Priors for the parameters of the single-component fits at 1.4, 2.4, and 3.4 days. $v_{\mathrm{outer}}$ is fixed to $0.35c$ in the 1.4 day fit (\citealt{vieira23}).}
\tablehead{parameter & 1.4 days & 2.4 days & 3.4 days}
\startdata\tablewidth{1.0\textwidth}
 \vspace{2pt}
$\log_{10}(L_\mathrm{outer}/L_{\odot})$ & $[7.6, 8.0]$ & $[7.2, 7.8]$ & TBD \\ 
$\log_{10}(\rho_0/\mathrm{g~cm^{-3}})$ & $[-16.0, -14.0]$ & same & same \\
$v_{\mathrm{inner}}/c$& $[0.250, 0.340]$ & $[0.100, 0.275]$ & TBD \\
$v_{\mathrm{outer}}/c$& - & $[0.280, 0.400]$ & TBD \\
$v_{\mathrm{exp}}/c$ & $[0.05, 0.30]$ & same & same \\
$Y_e$ & $[0.01, 0.40]$ & same & same \\
$s~[k_{\mathrm{B}}/\mathrm{nucleon}]$ & $[10, 35]$ & same & same \\
\enddata
\end{deluxetable}\label{tab:priors-single}


\begin{deluxetable}{c|c}
\centering
\tablecaption{Priors for the parameters of the multi-component fits at 1.4, 2.4, and 3.4 days.}
\tablehead{parameter & prior}
\startdata\tablewidth{1.0\textwidth}
 \vspace{2pt}
$\log_{10}(L_\mathrm{outer}/L_{\odot})$ & $[7.5, 8.0]$ \\ 
$\log_{10}(\rho_0/\mathrm{g~cm^{-3}})$ & $[-16.0, -14.0]$ \\\hline
$v_{\mathrm{inner,red}}/c$& $[0.10, 0.35]$ \\
$v_{\mathrm{outer,red}}/c$ &  $[0.25, 0.40]$ \\
$v_{\mathrm{exp,red}}/c$ & $[0.05, 0.30]$ \\
$Y_{e,\mathrm{red}}$ & $[0.01, 0.45]$ \\
$s_{\mathrm{red}}~[k_{\mathrm{B}}/\mathrm{nucleon}]$ & $[10, 35]$ \\\hline
$v_{\mathrm{inner,blu}}/c$& $[0.10, 0.35]$ \\
$v_{\mathrm{outer,blu}}/c$ &  $[0.25, 0.40]$ \\
$v_{\mathrm{exp,blu}}/c$ & $[0.05, 0.30]$ \\
$Y_{e,\mathrm{blu}}$ & $[0.05, 0.50]$ \\
$s_{\mathrm{blu}}~[k_{\mathrm{B}}/\mathrm{nucleon}]$ & $[10, 35]$ \\
\enddata
\end{deluxetable}\label{tab:priors-multi}


%%% === SECTION 3 === %%%
%% RESULTS %%

\section{Results}\label{sec:results}

Things not to forget:

\begin{itemize}

    \item In multi-component models, compare the mass of one component to another to determine if one ``dominates'' over another
    
    

\end{itemize}

\subsection{2.4 and 3.4 days, single-epoch}



\subsection{1.4, 2.4, and 3.4 days, multi-epoch} 

%%% === SECTION 4 === %%%
%% DISCUSSION %%

\section{Discussion}\label{sec:disco}

\subsection{Time evolution of the abundances}

\subsection{Single- versus multi-component}


%%% === SECTION 5 === %%%
%% CONCLUSIONS %%

\section{Conclusions}\label{sec:conclusions}



%%% === ACKNOWLEDGEMENTS === %%%
\acknowledgments


NV works in Tiohti{\'a}:ke / Mooniyang, also known as Montr{\'e}al, which lies on the unceded land of the Haudenosaunee and Anishinaabeg nations. This work made use of high-performance computing resources in Tiohti{\'a}:ke / Mooniyang and in Burnaby, British Columbia, the unceded land of the Coast Salish peoples, including the Tsleil-Waututh, Kwikwetlem, Squamish, and Musqueam nations. We acknowledge the ongoing struggle of Indigenous peoples on this land, and elsewhere on Turtle Island, and hope for a future marked by true reconciliation. 

This work made extensive use of the \href{https://docs.alliancecan.ca/wiki/Cedar}{\texttt{Cedar}} and \href{https://docs.alliancecan.ca/wiki/Narval/en}{\texttt{Narval}} clusters of the \href{https://alliancecan.ca/en}{Digital Research Alliance of Canada} at Simon Fraser University (with regional partner \href{https://www.westgrid.ca/}{WestGrid}) and the {\'E}cole de technologie sup{\'e}rieure, respectively. We thank the support staff of Calcul Qu{\'e}bec in particular for their assistance at various steps in this project. We also thank Nikolas Provatas, Victor Ionescu, and Bart Odelman.

We thank Shinya Wanajo for kindly sharing their reaction network calculations. We thank Jessica Birky and David Fleming for useful discussions on approximate Bayesian inference and the use of \href{https://dflemin3.github.io/approxposterior/index.html}{\approxposterior}.

This work has made use of the \href{http://vald.astro.uu.se/~vald/php/vald.php}{Vienna Atomic Line Database (VALD)}, operated at Uppsala University, the Institute of Astronomy RAS in Moscow, and the University of Vienna. We thank Nikolai Piskunov and Eric Stempels for help in obtaining the VALD data.

This research made use of \href{https://tardis-sn.github.io/tardis/index.html}{\TARDIS}, a community-developed software package for spectral synthesis in supernovae (\citealt{kerzendorf14}). The development of \TARDIS~received support from the Google Summer of Code initiative and from the European Space Agency (ESA)'s Summer of Code in Space program. \TARDIS~makes extensive use of \href{https://docs.astropy.org/en/stable/}{\texttt{astropy}}. We thank Andrew Fullard, Wolfgang Kerzendorf, and the entire \TARDIS~development team for their assistance and their commitment to the development and maintenance of the code. 

N.V. acknowledges funding from the Bob Wares Science Innovation Prospectors Fund and the Murata Family Fellowship. J.J.R.\ and D.H.\ acknowledge support from the Canada Research Chairs (CRC) program, the NSERC Discovery Grant program, the FRQNT Nouveaux Chercheurs Grant program, and the Canadian Institute for Advanced Research (CIFAR). J.J.R.\ acknowledges funding from the Canada Foundation for Innovation (CFI), and the Qu\'{e}bec Ministère de l’\'{E}conomie et de l’Innovation.
\newline

%%% === SOFTWARE === %%%
\software{
\href{https://dflemin3.github.io/approxposterior/index.html}{\approxposterior}: \cite{fleming18};
\href{https://docs.astropy.org/en/stable/}{\texttt{astropy}}: \cite{astropy18};
\href{https://cmasher.readthedocs.io/}{\texttt{cmasher}}: \cite{velden20};
\href{https://corner.readthedocs.io/en/latest/index.html}{\texttt{corner}}: \cite{foreman-mackey16};
\href{https://dynesty.readthedocs.io/en/latest/index.html}{\texttt{dynesty}}: 
\cite{speagle20};
% \href{https://emcee.readthedocs.io/en/stable/}{\texttt{emcee}}: \cite{foreman-mackey13}; 
\href{https://george.readthedocs.io/en/latest/}{\texttt{george}}: \cite{ambikasaran15};
\href{https://tardis-sn.github.io/tardis/index.html}{\TARDIS}: \cite{kerzendorf14};
% \href{https://johannesbuchner.github.io/UltraNest/}{\texttt{UltraNest}}: \cite{buchner21}
} 

\clearpage




%%% === BIBLIOGRAPHY === %%%
\bibliographystyle{apj}
\begin{thebibliography}{}


%\bibitem[Abbott \& Lucy(1985)]{abbott85} Abbott, D.~C. \& Lucy, L.~B.\ 1985, \apj, 288, 679
%%%%% TARDIS LUCY FORMALISM


% \bibitem[Abbott et al.(2017a)]{abbottLIGO17a} Abbott, B.~P., Abbott, R., Abbott, T.~D., et al.\ 2017a, \aj, 848, L12
% %%%%% G170817: MULTI-MESSENGER


% \bibitem[Abbott et al.(2017b)]{abbottLIGO17b} Abbott, B.~P., Abbott, R., Abbott, T.~D., et al.\ 2017b, \apjl, 848, L13
% %%%%% GW170817: GWs AND GRBs


% \bibitem[Abbott et al.(2017)]{abbott17c} Abbott, B.~P., Abbott, R., Abbott, T.~D., et al.\ 2017, \prl, 119, 161101
% %%%%% GW170817: GWs 

%\bibitem[Abbott et al.(2018)]{abbottLIGO18} Abbott, B.~P., Abbott, R., Abbott, T.~D., et al.\ 2018, Living Reviews in Relativity, 21, 3


%\bibitem[Abbott et al.(2018)]{abbott18} Abbott, T.~M.~C., Abdalla, F.~B., Allam, S., et al.\ 2018, \apjs, 239, 18


% \bibitem[Ai et al.(2022)]{ai22} Ai, S., Zhang, B., \& Zhu, Z.\ 2022, \mnras, 516, 2614


% \bibitem[Almualla et al.(2021)]{almualla21} Almualla, M., Ning, Y., Bulla, M., et al.\ 2021, arXiv:2112.1547
% %%%%% INFERENCE WITH POSSIS



\bibitem[Ambikasaran et al.(2015)]{ambikasaran15} Ambikasaran, S., Foreman-Mackey, D., Greengard, L., et al.\ 2015, IEEE Transactions on Pattern Analysis and Machine Intelligence, 38, 252


% \bibitem[Anand et al.(2021)]{anand21} Anand, S., Coughlin, M.~W., Kasliwal, M.~M., et al.\ 2021, Nature Astronomy, 5, 46
% %%%%% POSSIS DATASET USED IN LUKOSIUTE+22


% \bibitem[Andreoni et al.(2017)]{andreoni17} Andreoni, I., Ackley, K., Cooke, J., et al.\ 2017, \pasa, 34, e069
% %%%%% GW170817 OBSERVATIONS


% \bibitem[Arcavi et al.(2017)]{arcavi17} Arcavi, I., Hosseinzadeh, G., Howell, D.~A., et al.\ 2017, \nat, 551, 64
% %%%%% GW170817 OBSERVATIONS


%\bibitem[Arnett(1982)]{arnett82} Arnett, W.~D.\ 1982, \apj, 253, 785


\bibitem[Astropy Collaboration et al.(2018)]{astropy18} Astropy Collaboration, Price-Whelan, A.~M., Sip{\H{o}}cz, B.~M., et al.\ 2018, \aj, 156, 123


% \bibitem[Badnell(2016)]{badnell16} Badnell, N.~R.\ 2016, AUTOSTRUCTURE: General program for calculation of atomic and ionic properties, ascl:1612.014


% \bibitem[Baiotti \& Rezzolla(2017)]{baiotti17} Baiotti, L. \& Rezzolla, L.\ 2017, Reports on Progress in Physics, 80, 096901.


%\bibitem[Banerjee et al.(2020)]{banerjee20} Banerjee, P., Wu, M.-R., \& Yuan, Z.\ 2020, \apjl, 902, L34
%%%%% ASTROPHYSICAL SITE OF THE R-PROCESS


%\bibitem[Bar-Shalom et al.(2001)]{barshalom01} Bar-Shalom, A., Klapisch, M., \& Oreg, J.\ 2001, \jqsrt, 71, 169


%\bibitem[Barnes \& Kasen(2013)]{barnes13} Barnes, J., Kasen, D.\ 2013, \aj, 775, 18


% \bibitem[Barnes et al.(2016)]{barnes16} Barnes, J., Kasen, D., Wu, M., Mart\'{i}nez-Pinedo, G.\ 2016, \apj, 829, 110


% \bibitem[Barnes et al.(2021)]{barnes21} Barnes, J., Zhu, Y.~L., Lund, K.~A., et al.\ 2021, \apj, 918, 44


%\bibitem[Barstow \& Heng(2020)]{barstow20} Barstow, J.~K., \& Heng, K.\ 2020, arXiv e-prints, arXiv:2003.14311


% \bibitem[Bartos \& Marka(2019)]{bartos19} Bartos, I. \& Marka, S.\ 2019, \nat, 569, 85
% %%%%% ASTROPHYSICAL SITE OF THE R-PROCESS


% \bibitem[Bauswein et al.(2013)]{bauswein13} Bauswein, A., Goriely, S., \& Janka, H.-T.\ 2013, \apj, 773, 78



% \bibitem[Beniamini et al.(2016)]{beniamini16} Beniamini, P., Hotokezaka, K., \& Piran, T.\ 2016, \apj, 832, 149
% %%%%% ASTROPHYSICAL SITE OF THE R-PROCESS


%\bibitem[Bethe \& Brown(1998)]{bethe98} Bethe, H.~A., \& Brown, G.~E.\ 1998, \apj, 506, 780

% \bibitem[Bisterzo et al.(2014)]{bisterzo14} Bisterzo, S., Travaglio, C., Gallino, R., et al.\ 2014, \apj, 787, 10


% \bibitem[Brauer et al.(2020)]{brauer20} Brauer, K., Ji, A.~P., Drout, M.~R., et al.\ 2020, arXiv:2010.15837
% %%%%% ASTROPHYSICAL SITE OF THE R-PROCESS


% \bibitem[Buchner(2021)]{buchner21} Buchner, J.\ 2021, The Journal of Open Source Software, 6, 3001
% %%%%% ULTRANEST


% \bibitem[Bulla(2019)]{bulla19} Bulla, M.\ 2019, \mnras, 489, 5037


% \bibitem[Burbidge et al.(1957)]{burbidge57} Burbidge, E.~M., Burbidge, G.~R., Fowler, W.~A., et al.\ 1957, Reviews of Modern Physics, 29, 547


% \bibitem[Cameron(1957)]{cameron57} Cameron, A.~G.~W.\ 1957, \pasp, 69, 201


%\bibitem[Castor(1974)]{castor74} Castor, J.~L.\ 1974, \mnras, 169, 279
%%%%% EXPANSION OPACITY FORMALISM


% \bibitem[Chornock et al.(2017)]{chornock17} Chornock, R., Berger, E., Kasen, D., et al.\ 2017, \apjl, 848, L19


%\bibitem[Christie et al.(2019)]{christie19} Christie, I.~M., Lalakos, A., Tchekhovskoy, A., et al.\ 2019, \mnras, 490, 4811


% \bibitem[Ciolfi(2020)]{ciolfi20a} Ciolfi, R.\ 2020, General Relativity and Gravitation, 52, 59
% %%%%% IMPORTANCE OF B-FIELDS


% \bibitem[Ciolfi \& Kalinani(2020)]{ciolfi20b} Ciolfi, R. \& Kalinani, J.~V.\ 2020, \apjl, 900, L35
% %%%%% B-DRIVEN WIND FOR GW170817


% \bibitem[C{\^o}t{\'e} et al.(2018)]{cote18} C{\^o}t{\'e}, B., Fryer, C.~L., Belczynski, K., et al.\ 2018, \apj, 855, 99
% %%%%% ASTROPHYSICAL SITE OF THE R-PROCESS


% \bibitem[C{\^o}t{\'e} et al.(2019)]{cote19} C{\^o}t{\'e}, B., Eichler, M., Arcones, A., et al.\ 2019, \apj, 875, 106
% %%%%% ASTROPHYSICAL SITE OF THE R-PROCESS


% \bibitem[C{\^o}t{\'e} et al.(2021)]{cote21} C{\^o}t{\'e}, B., Eichler, M., Yag{\"u}e L{\'o}pez, A., et al.\ 2021, Science, 371, 945
% %%%%% ASTROPHYSICAL SITE OF THE R-PROCESS


% \bibitem[Coulter et al.(2017)]{coulter17} Coulter, D.~A., Foley, R.~J., Kilpatrick, C.~D., et al.\ 2017, Science, 358, 1556
% %%%%% GW170817 OBSERVATIONS


%\bibitem[Cowan \& Griffin(1976)]{cowan76} Cowan, R.~D. \& Griffin, D.~C.\ 1976, Journal of the Optical Society of America (1917-1983), 66, 1010
%%%%% AUTOSTRUCTURE CALCULATIONS


% \bibitem[Cowan et al.(2021)]{cowan21} Cowan, J.~J., Sneden, C., Lawler, J.~E., et al.\ 2021, Reviews of Modern Physics, 93, 015002
% %%%%% ASTROPHYSICAL SITE OF THE R-PROCESS


% \bibitem[Cunha et al.(2017)]{cunha17} Cunha, K., Smith, V.~V., Hasselquist, S., et al.\ 2017, \apj, 844, 145
% %%%% APOGEE


\bibitem[Czekala et al.(2015)]{czekala15} Czekala, I., Andrews, S.~M., Mandel, K.~S., et al.\ 2015, \apj, 812, 128


% \bibitem[Darbha \& Kasen(2020)]{darbha20} Darbha, S. \& Kasen, D.\ 2020, \apj, 897, 150


% \bibitem[D{\'\i}az et al.(2017)]{diaz17} D{\'\i}az, M.~C., Macri, L.~M., Garcia Lambas, D., et al.\ 2017, \apjl, 848, L29
% %%%%% GW170817 OBSERVATIONS


% \bibitem[Dietrich et al.(2020)]{dietrich20} Dietrich, T., Coughlin, M.~W., Pang, P.~T.~H., et al.\ 2020, Science, 370, 1450
% %%%%% POSSIS DATASET USED IN LUKOSIUTE+22


\bibitem[Domoto et al.(2021)]{domoto21} Domoto, N., Tanaka, M., Wanajo, S., et al.\ 2021, \apj, 913, 26


\bibitem[Domoto et al.(2022)]{domoto22} Domoto, N., Tanaka, M., Kato, D., et al.\ 2022, arXiv:2206.04232


% \bibitem[Drout et al.(2017)]{drout17} Drout, M.~R., Piro, A.~L., Shappee, B.~J., et al.\ 2017, Science, 358, 6370, 1570-1574
% %%%%% GW170817 OBSERVATIONS


%\bibitem[Eastman \& Pinto(1993)]{eastman93} Eastman, R.~G. \& Pinto, P.~A.\ 1993, \apj, 412, 731
%%%%% EXPANSION OPACITY FORMALISM


% \bibitem[Eichler et al.(1989)]{eichler89} Eichler, D., Livio, M., Piran, T., et al.\ 1989, \nat, 340, 126
% %%%%% HISTORICAL RE. NSMs


% \bibitem[Eichler et al.(2015)]{eichler15} Eichler, M., Arcones, A., Kelic, A., et al.\ 2015, \apj, 808, 30


% \bibitem[Eichler et al.(2019)]{eichler19} Eichler, M., Sayar, W., Arcones, A., et al.\ 2019, \apj, 879, 47
% %%%%% ASTROPHYSICAL SITE OF THE R-PROCESS


%\bibitem[Etienne et al.(2009)]{etienne09} Etienne, Z.~B., Liu, Y.~T., Shapiro, S.~L.\ 2009, \prd, 79, 044024


% \bibitem[Evans et al.(2017)]{evans17} Evans, P.~A., Cenko, S.~B., Kennea, J.~A., et al.\ 2017, Science, 358, 1565
% %%%%% GW170817 OBSERVATIONS


% \bibitem[Even et al.(2020)]{even20} Even, W., Korobkin, O., Fryer, C.~L., et al.\ 2020, \apj, 899, 24


% \bibitem[Fahlman \& Fern{\'a}ndez(2018)]{fahlman18} Fahlman, S. \& Fern{\'a}ndez, R.\ 2018, \apjl, 869, L3


%\bibitem[Fern{\'a}ndez \& Metzger(2013)]{fernandez13} Fern{\'a}ndez, R., \& Metzger, B.~D.\ 2013, \mnras, 435, 502


% \bibitem[Fern{\'a}ndez \& Metzger(2016)]{fernandez16} Fern{\'a}ndez, R. \& Metzger, B.~D.\ 2016, Annual Review of Nuclear and Particle Science, 66, 23


%\bibitem[Fern{\'a}ndez et al.(2017)]{fernandez17} Fern{\'a}ndez, R., Foucart, F., Kasen, D., et al.\ 2017, Classical and Quantum Gravity, 34, 154001


%\bibitem[Fern{\'a}ndez et al.(2019)]{fernandez19} Fern{\'a}ndez, R., Tchekhovskoy, A., Quataert, E., et al.\ 2019, \mnras, 482, 3373


\bibitem[Fleming \& VanderPlas(2018)]{fleming18} Fleming, D.~P., \& VanderPlas, J.\ 2018, The Journal of Open Source Software, 3, 781
%%%%% APPROXPOSTERIOR


\bibitem[Fleming et al.(2020)]{fleming20} Fleming, D.~P., Barnes, R., Luger, R., et al.\ 2020, \apj, 891, 155
%%%%% APPROXPOSTERIOR


% \bibitem[Fontes et al.(2020)]{fontes20} Fontes, C.~J., Fryer, C.~L., Hungerford, A.~L., et al.\ 2020, \mnras, 493, 4143


% \bibitem[Foreman-Mackey et al.(2013)]{foreman-mackey13} Foreman-Mackey, D., Hogg, D.~W., Lang, D., et al.\ 2013, \pasp, 125, 306
%%%%% EMCEE


\bibitem[Foreman-Mackey(2016)]{foreman-mackey16} Foreman-Mackey, D.\ 2016, The Journal of Open Source Software, 1, 24
%%%%% CORNER


%\bibitem[Foucart et al.(2014)]{foucart14} Foucart, F., Deaton, M.~B., Duez, M.~D., et al.\ 2014, \prd, 90, 024026


%\bibitem[Foucart et al.(2018)]{foucart18} Foucart, F., Hinderer, T., \& Nissanke, S.\ 2018, \prd, 98, 081501


%\bibitem[Foucart et al.(2019)]{foucart19} Foucart, F., Duez, M.~D., Kidder, L.~E., et al.\ 2019, \prd, 99, 103025

 
% \bibitem[Freiburghaus et al.(1999)]{freiburghaus99} Freiburghaus, C., Rosswog, S., \& Thielemann, F.-K.\ 1999, \apjl, 525, L121
% %%%%% HISTORICAL RE. NSMs


% \bibitem[Fujibayashi et al.(2020)]{fujibayashi20} Fujibayashi, S., Wanajo, S., Kiuchi, K., et al.\ 2020, \apj, 901, 122
% %%%%% POST-MERGER EJECTA FOR LOW-MASS BNSs


\bibitem[Gillanders et al.(2021)]{gillanders21} Gillanders, J.~H., McCann, M., Smartt, S.~A.~S.~S.~J., et al.\ 2021, arXiv:2101.08271
%%%%% GOLD AND PLATINUM IN GW170817


\bibitem[Gillanders et al.(2022)]{gillanders22} Gillanders, J.~H., Smartt, S.~J., Sim, S.~A., et al.\ 2022, arXiv:2202.01786
%%%%% MODELLING THE SPECTRUM


%\bibitem[Goriely(1999)]{goriely99} Goriely, S.\ 1999, \aap, 342, 881
%%%%% r-RESIDUALS


% \bibitem[Goriely et al.(2011)]{goriely11} Goriely, S., Bauswein, A., \& Janka, H.-T.\ 2011, \apjl, 738, L32


% \bibitem[Grossman et al.(2014)]{grossman14} Grossman, D., Korobkin, O., Rosswog, S., et al.\ 2014, \mnras, 439, 757


% \bibitem[Hasselquist et al.(2016)]{hasselquist16} Hasselquist, S., Shetrone, M., Cunha, K., et al.\ 2016, \apj, 833, 81
% %%%%% APOGEE


% \bibitem[Heinzel et al.(2021)]{heinzel21} Heinzel, J., Coughlin, M.~W., Dietrich, T., et al.\ 2021, \mnras, 502, 3057


% \bibitem[Holmbeck et al.(2018)]{holmbeck18} Holmbeck, E.~M., Beers, T.~C., Roederer, I.~U., et al.\ 2018, \apjl, 859, L24
% %%%%% ACTINIDE-BOOST STARS


% \bibitem[Holmbeck et al.(2019a)]{holmbeck19a} Holmbeck, E.~M., Sprouse, T.~M., Mumpower, M.~R., et al.\ 2019, \apj, 870, 23
% %%%%% ACTINIDE-BOOST; ASTROPHYSICAL SITE OF R-PROCESS


% \bibitem[Holmbeck et al.(2019b)]{holmbeck19b} Holmbeck, E.~M., Frebel, A., McLaughlin, G.~C., et al.\ 2019, \apj, 881, 5
% %%%%% ACTINIDE-BOOST; ASTROPHYSICAL SITE OF R-PROCESS


% \bibitem[Hotokezaka et al.(2021)]{hotokezaka21} Hotokezaka, K., Tanaka, M., Kato, D., et al.\ 2021, \mnras, 506, 5863
% %%%%% NEBULAR NSMs


% \bibitem[Hu et al.(2017)]{hu17} Hu, L., Wu, X., Andreoni, I., et al.\ 2017, Science Bulletin, 62, 1433
% %%%%% GW170817 OBSERVATIONS


%\bibitem[Ishimaru et al.(2015)]{ishimaru15} Ishimaru, Y., Wanajo, S., \& Prantzos, N.\ 2015, \apjl, 804, L35
%%%%% ASTROPHYSICAL SITE OF THE R-PROCESS


% \bibitem[Ji et al.(2016)]{ji16} Ji, A.~P., Frebel, A., Chiti, A., et al.\ 2016, \nat, 531, 610
% %%%%% ASTROPHYSICAL SITE OF THE R-PROCESS


% \bibitem[Ji et al.(2019)]{ji19} Ji, A.~P., Drout, M.~R., \& Hansen, T.~T.\ 2019, \apj, 882, 40


% \bibitem[Just et al.(2015)]{just15} Just, O., Bauswein, A., Ardevol Pulpillo, R., et al.\ 2015, \mnras, 448, 541


% \bibitem[Just et al.(2022)]{just22} Just, O., Kullmann, I., Goriely, S., et al.\ 2022, \mnras, 510, 2820


\bibitem[Kandasamy et al.(2017)]{kandasamy17} Kandasamy, K., Schneider, J., P{\'o}czos, B.\ 2017, Artificial Intelligence, 243


%\bibitem[Karp et al.(1977)]{karp77} Karp, A.~H., Lasher, G., Chan, K.~L., et al.\ 1977, \apj, 214, 161
%%%%% EXPANSION OPACITY FORMALISM


% \bibitem[Kasen et al.(2013)]{kasen13}Kasen, D., Badnell, N.~R., Barnes, J., \ 2013, \aj, 774, 25


% \bibitem[Kasen et al.(2015)]{kasen15} Kasen, D., Fern{\'a}ndez, R., \& Metzger, B.~D.\ 2015, \mnras, 450, 1777


\bibitem[Kasen et al.(2017)]{kasen17} Kasen, D., Metzger, B., Barnes, J., et al.\ 2017, \nat, 551, 80


%\bibitem[Kasen \& Barnes(2019)]{kasen19} Kasen, D., \& Barnes, J.\ 2019, \apj, 876, 128


% \bibitem[Kasliwal et al.(2017)]{kasliwal17} Kasliwal, M.~M., Nakar, E., Singer, L.~P., et al.\ 2017, Science, 358, 1559
% %%%%% GW170817 OBSERVATIONS


%\bibitem[Kasliwal et al.(2019)]{kasliwal19} Kasliwal, M.~M., Kasen, D., Lau, R.~M., et al.\ 2019, \mnras, L14


%\bibitem[Kawaguchi et al.(2015)]{kawaguchi15} Kawaguchi, K., Kyutoku, K., Nakano, H., et al.\ 2015, \prd, 92, 024014


%\bibitem[Kawaguchi et al.(2016)]{kawaguchi16} Kawaguchi, K., Kyutoku, K., Shibata, M., Tanaka, M.\ 2016, \aj, 825, 52


\bibitem[Kawaguchi et al.(2020)]{kawaguchi20} Kawaguchi, K., Shibata, M., \& Tanaka, M.\ 2020, \apj, 889, 171
%%%%% DIVERSITY OF KNe


\bibitem[Kerzendorf \& Sim(2014)]{kerzendorf14} Kerzendorf, W.~E., \& Sim, S.~A.\ 2014, \mnras, 440, 387


%\bibitem[Khatami \& Kasen(2019)]{khatami19} Khatami, D.~K., \& Kasen, D.~N.\ 2019, \apj, 878, 56


% \bibitem[Klion et al.(2022)]{klion22} Klion, H., Tchekhovskoy, A., Kasen, D., et al.\ 2022, \mnras, 510, 2968


% \bibitem[Korobkin et al.(2012)]{korobkin12} Korobkin, O., Rosswog, S., Arcones, A., Winteler, C., et al.\ 2012, \mnras, 426, 3, 1940-1949


% \bibitem[Korobkin et al.(2021)]{korobkin21} Korobkin, O., Wollaeger, R.~T., Fryer, C.~L., et al.\ 2021, \apj, 910, 116


% \bibitem[Kramida et al.(2019)]{kramida19} Kramida, A., Ralchenko, Y., Reader, J., and {NIST ASD Team} \ 2019, NIST Atomic Spectra Database (ver 5.7.1), National Institute of Standards and Technology


% \bibitem[Kullmann et al.(2022a)]{kullmann22a} Kullmann, I., Goriely, S., Just, O., et al.\ 2022, \mnras, 510, 2804
% %%%%% DYNAMICAL EJECTA WITH WEAK PROCESSES


% \bibitem[Kullmann et al.(2022b)]{kullmann22b} Kullmann, I., Goriely, S., Just, O., et al.\ 2022, arXiv:2207.07421
% %%%% IMPACT OF NUCLEAR UNCERTAINTIES


%\bibitem[Kurucz \& Bell(1995)]{kurucz95} Kurucz, R., \& Bell, B.\ 1995, Atomic Line Data (R.L. Kurucz and B. Bell) Kurucz CD-ROM No. 23. Cambridge, 23


% \bibitem[Kurucz(2018)]{kurucz18} Kurucz, R.~L.\ 2018, Workshop on Astrophysical Opacities, 515, 47

%\bibitem[Kyutoku et al.(2015)]{kyutoku15} Kyutoku, K., Ioka, K., Okawa, H., et al.\ 2015, \prd, 92, 044028


%\bibitem[Kyutoku et al.(2018)]{kyutoku18} Kyutoku, K., Kiuchi, K., Sekiguchi, Y., et al.\ 2018, \prd, 97, 023009


% \bibitem[Lai et al.(2008)]{lai08} Lai, D.~K., Bolte, M., Johnson, J.~A., et al.\ 2008, \apj, 681, 1524
% %%%%% ACTINIDE-BOOST STARS


% \bibitem[Lattimer \& Schramm(1974)]{lattimer74} Lattimer, J.~M., \& Schramm, D.~N.\ 1974, \apjl, 192, L145


%\bibitem[Li \& Paczy{\'n}ski(1998)]{li98} Li, L.-X., \& Paczy{\'n}ski, B.\ 1998, \apjl, 507, L59


% \bibitem[Lippuner \& Roberts(2015)]{lippuner15} Lippuner, J. \& Roberts, L.~F.\ 2015, \apj, 815, 82


% \bibitem[Lippuner et al.(2017)]{lippuner17} Lippuner, J., Fern\'{a}ndez, R., Roberts, L.~F., et al.\ 2017, \mnras, 472, 1, 904-918


% \bibitem[Lipunov et al.(2017)]{lipunov17} Lipunov, V.~M., Gorbovskoy, E., Kornilov, V.~G., et al.\ 2017, \apjl, 850, L1
% %%%%% GW170817 OBSERVATIONS


% \bibitem[Lodders et al.(2009)]{lodders09} Lodders, K., Palme, H., \& Gail, H.-P.\ 2009, Landolt B\&ouml;rnstein, 4B, 712

% \bibitem[Long \& Knigge(2002)]{long02} Long, K.~S. \& Knigge, C.\ 2002, \apj, 579, 725
% %%%%% TARDIS STUFF: VIRTUAL PACKETS


%\bibitem[Lovelace et al.(2013)]{lovelace13} Lovelace, G., Duez, M.~D., Foucart, F., et al.\ 2013, Class. Quant. Grav., 30, 13


% \bibitem[Lucy(1999)]{lucy99} Lucy, L.~B.\ 1999, \aap, 344, 282
% %%%%% TARDIS LUCY FORMALISM


%\bibitem[Lucy(1999b)]{lucy99b} Lucy, L.~B.\ 1999, \aap, 345, 211
%%%%% TARDIS LUCY FORMALISM


% \bibitem[Lucy(2002)]{lucy02} Lucy, L.~B.\ 2002, \aap, 384, 725
% %%%%% TARDIS LUCY FORMALISM: MACROATOM


% \bibitem[Lucy(2003)]{lucy03} Lucy, L.~B.\ 2003, \aap, 403, 261
% %%%%% TARDIS LUCY FORMALISM: MORE RE. MACROATOM


%\bibitem[Lucy(2005)]{lucy05} Lucy, L.~B.\ 2005, \aap, 429, 19
%%%%% TARDIS LUCY FORMALISM


% \bibitem[Luko{\v{s}}iute et al.(2022)]{lukosiute22} Luko{\v{s}}iute, K., Raaijmakers, G., Doctor, Z., et al.\ 2022, \mnras, 516, 1137


% \bibitem[Majewski et al.(2017)]{majewski17} Majewski, S.~R., Schiavon, R.~P., Frinchaboy, P.~M., et al.\ 2017, \aj, 154, 94
% %%%%% APOGEE

% \bibitem[Margutti \& Chornock(2021)]{margutti21} Margutti, R. \& Chornock, R.\ 2021, \araa, 59, 155

% \bibitem[Mazzali \& Lucy(1993)]{mazzali93} Mazzali, P.~A. \& Lucy, L.~B.\ 1993, \aap, 279, 447
% %%%%% TARDIS LUCY FORMALISM


% \bibitem[Mendoza-Temis et al.(2015)]{mendoza-temis15} Mendoza-Temis, J. de J., Wu, M.-R., Langanke, K., et al.\ 2015, \prc, 92, 055805


%\bibitem[Metzger et al.(2008)]{metzger08} Metzger, B.~D., Piro, A.~L., \& Quataert, E.\ 2008, \mnras, 390, 781


% \bibitem[Metzger et al.(2010)]{metzger10} Metzger, B.~D., Arcones, A., Quataert, E., et al.\ 2010, \mnras, 402, 2771


%\bibitem[Metzger et al.(2010)]{metzger10} Metzger, B.~D., Mart{\'i}nez-Pinedo, G., Darbha, S., et al.\ 2010, \mnras, 406, 4, 2650-2662


% \bibitem[Metzger \& Fern{\'a}ndez(2014)]{metzger14} Metzger, B.~D., \& Fern{\'a}ndez, R.\ 2014, \mnras, 441, 3444


% \bibitem[Metzger et al.(2018)]{metzger18} Metzger, B.~D., Thompson, T.~A., \& Quataert, E.\ 2018, \apj, 856, 101


% \bibitem[Metzger(2019)]{metzger19} Metzger, B.~D.\ 2019, Living Rev Relativ, 23, 1
% %%%%% TOME ON KNe


% \bibitem[Miller et al.(2019)]{miller19} Miller, J.~M., Ryan, B.~R., Dolence, J.~C., et al.\ 2019, \prd, 100, 023008. doi:10.1103/PhysRevD.100.023008


% \bibitem[Mumpower et al.(2016)]{mumpower16} Mumpower, M.~R., Surman, R., McLaughlin, G.~C., et al.\ 2016, Progress in Particle and Nuclear Physics, 86, 86


% \bibitem[Nativi et al.(2021)]{nativi21} Nativi, L., Bulla, M., Rosswog, S., et al.\ 2021, \mnras, 500, 1772


% \bibitem[Nedora et al.(2021)]{nedora21} Nedora, V., Bernuzzi, S., Radice, D., et al.\ 2021, \apj, 906, 98


\bibitem[Nelder \& Mead(1965)]{neldermead65} Nelder, J.~A., Mead, R.\ 1965, Computer Journal, 7, 308
%%%%% APPROXPOSTERIOR OPTIMIZER


% \bibitem[Noebauer \& Sim(2019)]{noebauer19} Noebauer, U.~M. \& Sim, S.~A.\ 2019, Living Reviews in Computational Astrophysics, 5, 1
% %%%%% REVIEW OF MONTE CARLO RADIATIVE TRANSFER



\bibitem[Pakhomov et al.(2019)]{pakhomov19} Pakhomov, Y.~V., Ryabchikova, T.~A., \& Piskunov, N.~E.\ 2019, Astronomy Reports, 63, 1010
%%%%% VALD 


%\bibitem[Palmeri et al.(2000)]{palmeri00} Palmeri, P., Quinet, P., Wyart, J.-F., et al.\ 2000, \physscr, 61, 323
%%%%% AUTOSTRUCTURE TECHNIQUES


% \bibitem[Perego et al.(2014)]{perego14} Perego, A., Rosswog, S., Cabez{\'o}n, R.~M., et al.\ 2014, \mnras, 443, 3134


% \bibitem[Perego et al.(2022)]{perego22} Perego, A., Vescovi, D., Fiore, A., et al.\ 2022, \apj, 925, 22
% %%%%% light elements in KNe


\bibitem[Petelet et al.(2009)]{petelet09} Petelet, M., Iooss, B., Asserin, O., et al.\ 2009, arXiv:0909.0329


\bibitem[Pian et al.(2017)]{pian17} Pian, E., D'Avanzo, P., Benetti, S., et al.\ 2017, \nat, 551, 67
%%%%% GW170817 SPECTRA


%\bibitem[Planck Collaboration et al.(2016)]{planck16} Planck Collaboration, Ade, P.~A.~R., Aghanim, N., et al.\ 2016, \aap, 594, A13
%%%%% LAMBDA-CDM COSMOLOGY


% \bibitem[Pognan et al.(2022a)]{pognan22a} Pognan, Q., Jerkstrand, A., \& Grumer, J.\ 2022, \mnras, 510, 3806
% %%%%% NEBULAR/NLTE KNe, 1/2

% \bibitem[Pognan et al.(2022b)]{pognan22b} Pognan, Q., Jerkstrand, A., \& Grumer, J.\ 2022, \mnras, 513, 5174
% %%%%% NEBULAR/NLTE KNe, 2/2


% \bibitem[Powell(1964)]{powell64} Powell, M.~J.~D.\ 1964, Computer Journal, 7, 155
% %%%%% APPROXPOSTERIOR OPTIMIZER


%\bibitem[Pozanenko et al.(2018)]{pozanenko18} Pozanenko, A.~S., Barkov, M.~V., Minaev, P.~Y., et al.\ 2018, \apjl, 852, L30
%%%%% GW170817 OBSERVATIONS


%\bibitem[Prantzos et al.(2020)]{prantzos20} Prantzos, N., Abia, C., Cristallo, S., et al.\ 2020, \mnras, 491, 1832
%%%%% r-RESIDUALS


% \bibitem[Radice et al.(2020)]{radice20} Radice, D., Bernuzzi, S., \& Perego, A.\ 2020, Annual Review of Nuclear and Particle Science, 70, 95


% \bibitem[Ristic et al.(2022)]{ristic22} Ristic, M., Champion, E., O'Shaughnessy, R., et al.\ 2022, Physical Review Research, 4, 013046
% %%%%% INFERENCE WITH POSSIS


% \bibitem[Roederer et al.(2009)]{roederer09} Roederer, I.~U., Kratz, K.-L., Frebel, A., et al.\ 2009, \apj, 698, 1963
% %%%%% ACTINIDE-BOOST STARS


% \bibitem[Roederer et al.(2016)]{roederer16} Roederer, I.~U., Mateo, M., Bailey, J.~I., et al.\ 2016, \aj, 151, 82


%\bibitem[Rosswog(2005)]{rosswog05} Rosswog, S.\ 2005, \aj, 634, 1202


% \bibitem[Rosswog et al.(2014)]{rosswog14} Rosswog, S., Korobkin, O., Arcones, A., et al.\ 2014, \mnras, 439, 744


%\bibitem[Rosswog et al.(2018)]{rosswog18} Rosswog, S., Sollerman, J., Feindt, U., et al.\ 2018, \aap, 615, A132


%\bibitem[Ruiz et al.(2018)]{ruiz18} Ruiz, M., Shapiro, S.~L., \& Tsokaros, A.\ 2018, \prd, 98, 123017


\bibitem[Ryabchikova et al.(2015)]{ryabchikova15} Ryabchikova, T., Piskunov, N., Kurucz, R.~L., et al.\ 2015, \physscr, 90, 054005
%%%%% VALD


%\bibitem[Sagu{\'e}s Carracedo et al.(2021)]{sagues21} Sagu{\'e}s Carracedo, A., Bulla, M., Feindt, U., et al.\ 2021, \mnras, 504, 1294


\bibitem[Savitzky \& Golay(1964)]{savitzky64} Savitzky, A. \& Golay, M.~J.~E.\ 1964, Analytical Chemistry, 36, 1627


%\bibitem[Schlafly \& Finkbeiner(2011)]{schlafly11} Schlafly, E.~F., \& Finkbeiner, D.~P.\ 2011, \apj, 737, 103
%%%%% COMPUTING EXTINCTIONS


%\bibitem[Scolnic et al.(2018)]{scolnic18} Scolnic, D., Kessler, R., Brout, D., et al.\ 2018, \apjl, 852, L3


%\bibitem[Setzer et al.(2019)]{setzer19} Setzer, C.~N., Biswas, R., Peiris, H.~V., et al.\ 2019, \mnras, 485, 4260


% \bibitem[Shappee et al.(2017)]{shappee17} Shappee, B.~J., Simon, J.~D., Drout, M.~R., et al.\ 2017, Science, 358, 1574
% %%%%% GW170817 OBSERVATIONS


%\bibitem[Shen et al.(2015)]{shen15} Shen, S., Cooke, R.~J., Ramirez-Ruiz, E., et al.\ 2015, \apj, 807, 115
%%%%% ASTROPHYSICAL SITE OF THE R-PROCESS


%\bibitem[Shibata \& Taniguchi(2008)]{shibata08} Shibata M., \& Taniguchi, K.\ 2008, \prd, 77, 084015


% \bibitem[Shibata \& Hotokezaka(2019)]{shibata19} Shibata, M. \& Hotokezaka, K.\ 2019, Annual Review of Nuclear and Particle Science, 69, 41
% %%%%% REVIEW OF EJECTION MECHANISMS


%\bibitem[Siegel \& Metzger(2017)]{siegel17} Siegel, D.~M., \& Metzger, B.~D.\ 2017, \prl, 119, 231102


% \bibitem[Siegel \& Metzger(2018)]{siegel18} Siegel, D.~M. \& Metzger, B.~D.\ 2018, \apj, 858, 52


% \bibitem[Siegel et al.(2019)]{siegel19} Siegel, D.~M., Barnes, J., \& Metzger, B.~D.\ 2019, \nat, 569, 241
% %%%%% ASTROPHYSICAL SITE OF THE R-PROCESS


\bibitem[Smartt et al.(2017)]{smartt17} Smartt, S.~J., Chen, T.-W., Jerkstrand, A., et al.\ 2017, \nat, 551, 75
%%%%% GW170817 SPECTRA, Cs AND Te


% \bibitem[Silva et al.(2022)]{silva22} Silva, R.~F., Sampaio, J.~M., Amaro, P., et al.\ 2022, Atoms, 10, 18
% %%%%% Nd III and U III opacities for KNe

%\bibitem[Sneden et al.(2008)]{sneden08} Sneden, C., Cowan, J.~J., \& Gallino, R.\ 2008, \araa, 46, 241
%%%%% r-RESIDUALS


%\bibitem[Sobolev(1960)]{sobolev60} Sobolev, V.~V.\ 1960, Cambridge: Harvard University Press, 1960
%%%%% EXPANSION OPACITY FORMALISM, SOBOLEV APPROXIMATION


\bibitem[Speagle(2020)]{speagle20} Speagle, J.~S.\ 2020, \mnras, 493, 3132
%%%%% dynesty: dynamic nested sampling

% \bibitem[Symbalisty \& Schramm(1982)]{symbalisty82} Symbalisty, E. \& Schramm, D.~N.\ 1982, \aplett, 22, 143


%\bibitem[Tanaka \& Hotokezaka(2013)]{tanaka13} Tanaka, M. \& Hotokezaka, K.\ 2013, \aj, 775, 113


%\bibitem[Tanaka et al.(2014)]{tanaka14} Tanaka, M., Hotokezaka, K., Kyutoku, K., et al.\ 2014, \aj, 780, 31


% \bibitem[Tanaka et al.(2017)]{tanaka17} Tanaka, M., Utsumi, Y., Mazzali, P.~A., et al.\ 2017, \pasj, 69, 102


%\bibitem[Tanaka et al.(2018)]{tanaka18} Tanaka, M., Kato, D., Gaigalas, G., et al.\ 2018, \aj, 852, 109


% \bibitem[Tanaka et al.(2020)]{tanaka20} Tanaka, M., Kato, D., Gaigalas, G., et al.\ 2020, \mnras
% %%%%% SYSTEMATIC OPACITY CALCULATIONS


% \bibitem[Tanvir et al.(2017)]{tanvir17} Tanvir, N.~R., Levan, A.~J., Gonz{\'a}lez-Fern{\'a}ndez, C., et al.\ 2017, \apjl, 848, L27
% %%%%% GW170817 OBSERVATIONS


%\bibitem[Tarumi et al.(2021)]{tarumi21} Tarumi, Y., Hotokezaka, K., \& Beniamini, P.\ 2021, arXiv:2102.03368
%%%%% ASTROPHYSICAL SITE OF THE R-PROCESS


% \bibitem[Troja et al.(2017)]{troja17} Troja, E., Piro, L., van Eerten, H., et al.\ 2017, \nat, 551, 71
% %%%%% GW170817 OBSERVATIONS


% \bibitem[Utsumi et al.(2017)]{utsumi17} Utsumi, Y., Tanaka, M., Tominaga, N., et al.\ 2017, \pasj, 69, 101
% %%%%% GW170817 OBSERVATIONS


% \bibitem[Valenti et al.(2017)]{valenti17} Valenti, S., Sand, D.~J., Yang, S., et al.\ 2017, \apjl, 848, L24
% %%%%% GW170817 OBSERVATIONS


\bibitem[van der Velden(2020)]{velden20} van der Velden, E.\ 2020, The Journal of Open Source Software, 5, 2004
%%%%% CMASHER FOR PLOTTING


% \bibitem[Vassh et al.(2021)]{vassh21} Vassh, N., McLaughlin, G.~C., Mumpower, M.~R., et al.\ 2021, \apj, 907, 98
% %%%%% MCMC TO PROBE r-PROCESS DYNAMICS


\bibitem[Vieira et al.(2023)]{vieira23} Vieira, N., Ruan, J.~J., Haggard, D., et al.\ 2023, \apj, 944, 123
%%%% SPARK PAPER 1

\bibitem[Villar et al.(2017)]{villar17}Villar, V.~A., Guillochon, J., Berger, E., et al.\ 2017 \aj, 851, L21
%%%%% COMPILATION OF GW170817 PHOTOMETRY


%\bibitem[Villar et al.(2018)]{villar18} Villar, V.~A., Cowperthwaite, P.~S., Berger, E., et al.\ 2018, \apjl, 862, L11


% \bibitem[Vogl et al.(2019)]{vogl19} Vogl, C., Sim, S.~A., Noebauer, U.~M., et al.\ 2019, \aap, 621, A29
% %%%%% FULL RELATIVITY IN TARDIS 


% \bibitem[Vogl et al.(2020)]{vogl20} Vogl, C., Kerzendorf, W.~E., Sim, S.~A., et al.\ 2020, \aap, 633, A88
% %%%%% USING TARDIS TO BUILD AN EMULATOR FOR SNe II


% \bibitem[Wallner et al.(2015)]{wallner15} Wallner, A., Faestermann, T., Feige, J., et al.\ 2015, Nature Communications, 6, 5956
% %%%%% ASTROPHYSICAL SITE OF THE R-PROCESS


% \bibitem[Wanajo et al.(2014)]{wanajo14} Wanajo, S., Sekiguchi, Y., Nishimura, N., et al.\ 2014, \apjl, 789, L39


\bibitem[Wanajo(2018)]{wanajo18} Wanajo, S.\ 2018, \apj, 868, 65
%%%%% OUR SOURCE OF ABUNDANCES


% \bibitem[Wanajo et al.(2021)]{wanajo21} Wanajo, S., Hirai, Y., \& Prantzos, N.\ 2021, arXiv:2106.03707


%\bibitem[Wang \& Li(2018)]{wang18} Wang, H. \& Li, J.\ 2018, Neural Computation, 30, 11, 3072-3094


\bibitem[Watson et al.(2019)]{watson19} Watson, D., Hansen, C.~J., Selsing, J., et al.\ 2019, \nat, 574, 497


% \bibitem[Wollaeger et al.(2018)]{wollaeger18} Wollaeger, R.~T., Korobkin, O., Fontes, C.~J., et al.\ 2018, \mnras, 478, 3298


% \bibitem[Wollaeger et al.(2021)]{wollaeger21} Wollaeger, R.~T., Fryer, C.~L., Chase, E.~A., et al.\ 2021, arXiv:2105.11543


% \bibitem[Wu et al.(2016)]{wu16} Wu, M.-R., Fern{\'a}ndez, R., Mart{\'\i}nez-Pinedo, G., et al.\ 2016, \mnras, 463, 2323


%\bibitem[Wu et al.(2019)]{wu19} Wu, M.-R., Barnes, J., Mart{\'\i}nez-Pinedo, G., et al.\ 2019, \prl, 122, 062701


% \bibitem[Yamazaki et al.(2021)]{yamazaki21} Yamazaki, Y., Kajino, T., Mathews, G.~J., et al.\ 2021, arXiv:2102.05891
% %%%%% ASTROPHYSICAL SITE OF THE R-PROCESS


% \bibitem[Yaron \& Gal-Yam(2012)]{yaron12} Yaron, O., \& Gal-Yam, A.\ 2012, \pasp, 124, 668
% %%%%% WISeREP


%\bibitem[Zhu et al.(2018)]{zhu18} Zhu, Y., Wollaeger, R.~T., Vassh, N., et al.\ 2018, \apjl, 863, L23


% \bibitem[Zhu et al.(2021)]{zhu21} Zhu, Y.~L., Lund, K.~A., Barnes, J., et al.\ 2021, \apj, 906, 94


\end{thebibliography}


\appendix{}

%%% === APPENDIX A === %%%
%% ABUNDANCES %%
\section{Detailed Ion Abundances}\label{app:ion_abunds}



% ============================

\end{document}
